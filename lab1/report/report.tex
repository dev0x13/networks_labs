\documentclass{report}
\usepackage[T1,T2A]{fontenc}
\usepackage[utf8]{inputenc}
\usepackage[russian]{babel}
\usepackage[a4paper, margin=1in]{geometry}
\usepackage{parcolumns}
\usepackage{fancyvrb}
\usepackage{hyperref}
\usepackage{unicode-math}
\usepackage{graphicx, float}
\usepackage{caption}


\title{
Отчет по лабораторной работе №1.\\
Реализация протоколов автоматического запроса повторной передачи Go-Back-N и Selective Repeat
}
\author{Мануилов Георгий, студент гр. 3640102/80201}
\date{}

\begin{document}

\maketitle

\renewcommand\thesection{\arabic{section}}

\section{Постановка задачи}
Требуется разработать систему из двух агентов, способных обмениваться данными друг с другом. Требования к системе:
\begin{itemize}
    \item Должна моделироваться ненадежность канала связи: с заданной вероятностью пакеты при передаче должны теряться.
    \item Должна обеспечиваться доставка получателю всех отправленных данных посредством протоколов автоматического запроса повторной передачи Go-Back-N и Selective Repeat.
\end{itemize}

\section{Технические подробности реализации}
Программа-агент реализована на языке программирования C++. Программа может функционировать в двух режимах: прием и передача. Система представляет из себя два экземпляра программы, запущенных в разных режимах. Взаимодействие между процессами осуществляется с помощью очередей сообщений посредством библиотеки Boost.Interprocess.
\newline
\newline
Программа разделена на следующие сущности:
\begin{itemize}
    \item {\bf OneWayTransducer} - примитивный приемник/передатчик. Не контролирует доставку, работает только в одну сторону, утилизирует одну очередь сообщений.
    \item {\bf SendingTransducer} - передатчик, реализующий протокол автоматического запроса повторной передачи данных, утилизирует две очереди сообщений (два экземпляра OneWayTransducer): одну для отправки, одну для получения подтверждения доставки.
    \item {\bf ReceivingTransducer} - приемник, реализующий протокол автоматического запроса повторной передачи данных, утилизирует две очереди сообщений (два экземпляра OneWayTransducer): одну для получения, одну для отправки подтверждения доставки.
\end{itemize}
Каждый пакет (сообщение) для передачи содержит порядковый номер и 1 байт данных.
\newline
Программа-агент получает на вход путь к конфигурационному файлу, который содержит в себе следующие поля:
\begin{itemize}
    \item {\bf mode} - режим функционирования агента (SEND/RECEIVE)
    \item {\bf mq\_id} - ID канала передачи данных (должен совпадать у приемника и передатчика)
    \item {\bf protocol} - ARQ протокол (GBN/SR)
    \item {\bf file} - путь к файлу с данными для передачи (только для передатчика)
    \item {\bf window\_size} - размер скользящего окна (только для передатчика)
    \item {\bf timeout} - время в милисекундах, после которого сообщение считается потерянным, если не пришло подтверждение доставки (только для передатчика)
    \item {\bf loss\_probability} - вероятность (от 0 до 1) потери сообщения (только для приемника)
\end{itemize}
\newline
Исходный код решения, а также инструкции по сборке и запуску представлены в git репозитории по ссылке \url{https://github.com/dev0x13/networks_labs/tree/master/lab1}.
\newpage
\section{Пример работы программы}
Конфигурация агента-передатчика:
\begin{Verbatim}[frame=single]
mode = SEND
mq_id = 1
protocol = GBN
file = /home/george/networks_labs/lab1/cfg/test_file
window_size = 2
timeout = 500
\end{Verbatim}
\newline
Конфигурация агента-приемника:
\begin{Verbatim}[frame=single]
mode = RECEIVE
mq_id = 1
protocol = GBN
loss_probability = 0.3
\end{Verbatim}
\newline
Вывод:
\begin{parcolumns}{2}
\colchunk[1]{
\begin{verbatim}
$ ./agent ../cfg/sender.cfg 
Attached to sending queue `1_receive`
Attached to receiving queue `1_ack`
Sent: Message[b: h] to `1_receive`
Sent: Message[b: e] to `1_receive`
Sent: Message[b: h] to `1_receive`
Sent: Message[b: e] to `1_receive`
Ack: Message[b: h] from `1_ack`
Sent: Message[b: e] to `1_receive`
Sent: Message[b: l] to `1_receive`
Ack: Message[b: e] from `1_ack`
Ack: Message[b: l] from `1_ack`
Sent: Message[b: l] to `1_receive`
Sent: Message[b: o] to `1_receive`
Ack: Message[b: l] from `1_ack`
Sent: Message[b: o] to `1_receive`
Sent: Message[b: 
] to `1_receive`
Ack: Message[b: o] from `1_ack`
Ack: Message[b: 
] from `1_ack`
Transmission finished!
Sent: hello
\end{verbatim}
}
\colchunk[2]{
\begin{verbatim}
$ ./agent ../cfg/receiver.cfg 
Attached to sending queue `1_ack`
Attached to receiving queue `1_receive`
Lost: Message[b: h] from `1_receive`
Received: Message[b: h] from `1_receive`
Lost: Message[b: e] from `1_receive`
Received: Message[b: e] from `1_receive`
Received: Message[b: l] from `1_receive`
Received: Message[b: l] from `1_receive`
Lost: Message[b: o] from `1_receive`
Received: Message[b: o] from `1_receive`
Received: Message[b: 
] from `1_receive`
Transmission finished!
Received: hello
\end{verbatim}
}
\end{parcolumns}

\newpage
\section{Оценка и сравнение эффективности протоколов}
Эффективность протоколов оценивалась по двум параметрам:
\begin{itemize}
\item по коэффициенту эффективности $k = \frac{\mbox{количество пакетов для передачи}}{\mbox{количество переданных пакетов}}$
\item по времени от начала до конца передачи в милисекундах $t$.
\end{itemize}
Для оценки проводилась серия сеансов передачи с разными размерами скользящего окна $w$ и с разными вероятностями потери пакетов $p$. Во всех сеансах использовался один и тот же набор данных (85 пакетов). Полученные зависимости представлены в таблицах 1 и 2, и на рис. 1-4.
\begin{table}[h]
\centering
\begin{tabular}{ |c|c|c|c|c| }
 \hline
 $p$ & $t_{GBN}$, мс & $k_{GBN}$ & $t_{SR}$, мс & $k_{SR}$ \\ [0.5ex] 
 \hline
 0 & 293 & 1.00 & 292 & 1.00 \\ 
 0.1 & 354 & 0.83 & 274 & 0.89 \\ 
 0.2 & 393 & 0.75 & 456 & 0.73 \\ 
 0.3 & 570 & 0.53 & 524 & 0.71 \\ 
 0.4 & 708 & 0.42 & 607 & 0.63 \\ 
 0.5 & 1081 & 0.27 & 768 & 0.51 \\ 
 0.6 & 1525 & 0.19 & 1234 & 0.34 \\ 
 0.7 & 1690 & 0.18 & 1337 & 0.32 \\ 
 0.8 & 3203 & 0.09 & 2052 & 0.21 \\ 
 0.9 & 7372 & 0.04 & 4922 & 0.09 \\ 
 \hline
\end{tabular}
\captionsetup{justification=centering}
\caption{Зависимость эффективности протоколов от вероятности потери пакета $p$ при фиксированном размере скользящего окна $w=3$}
\end{table}

\begin{table}[h]
\centering
\begin{tabular}{ |c|c|c|c|c| }
 \hline
 $w$ & $t_{GBN}$, мс & $k_{GBN}$ & $t_{SR}$, мс & $k_{SR}$ \\ [0.5ex] 
 \hline
 1 & 1718 & 0.5 & 1676 & 0.51 \\ 
 2 & 992 & 0.44 & 979 & 0.52 \\ 
 3 & 1097 & 0.27 & 697 & 0.53 \\ 
 4 & 913 & 0.25 & 739 & 0.44 \\ 
 5 & 1124 & 0.16 & 533 & 0.54 \\ 
 6 & 945 & 0.16 & 428 & 0.54 \\ 
 7 & 853 & 0.15 & 394 & 0.54 \\ 
 8 & 846 & 0.14 & 460 & 0.47 \\ 
 9 & 631 & 0.17 & 425 & 0.47 \\ 
 10 & 630 & 0.15 & 375 & 0.5 \\ 
 \hline
\end{tabular}
\captionsetup{justification=centering}
\caption{Зависимость эффективности протоколов от размера скользящего окна $w$ при фиксированной вероятности потери пакета $p=0.5$}
\end{table}

\begin{figure}[bp!]
    \centering
    \includegraphics[width=0.8\textwidth]{1.png}
    \caption{Зависимость коэффициента эффективности от вероятности потери пакета}
    \label{fig:1.png}
\end{figure}

\begin{figure}[bp!]
    \centering
    \includegraphics[width=0.8\textwidth]{2.png}
    \caption{Зависимость времени передачи от вероятности потери пакета}
    \label{fig:2.png}
\end{figure}

\begin{figure}[bp!]
    \centering
    \includegraphics[width=0.8\textwidth]{3.png}
    \caption{Зависимость коэффициента эффективности от размера скользящего окна}
    \label{fig:1.png}
\end{figure}

\begin{figure}[bp!]
    \centering
    \includegraphics[width=0.8\textwidth]{4.png}
    \caption{Зависимость времени передачи от размера скользящего окна}
    \label{fig:2.png}
\end{figure}

\newpage
По представленным зависимостям можно сделать следующие выводы:
\begin{itemize}
    \item При малых ($<0.3$) вероятностях потери пакета и фиксированном размере окна протоклы показывают одинковую эффективность.
    \item При больших ($>0.3$) вероятностях потери пакета и фиксированном размере окна Selective Repeat показывает большую эффективность, чем Go-Back-N.
    \item Размер окна не оказывает влияние на коэффициент эффективности для Selective Repeat при фиксированной вероятности потери пакета, но его увеличение ведет к уменьшению времени передачи, то есть чем больше размер окна, тем больше эффективность протокола.
    \item Увеличение размера окна при фиксированной вероятности потери пакета ведет к уменьшению эффективности Go-Back-N.
\end{itemize}

\end{document}

