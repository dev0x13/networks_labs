\documentclass{report}
\usepackage[T1,T2A]{fontenc}
\usepackage[utf8]{inputenc}
\usepackage[russian]{babel}
\usepackage[a4paper, margin=1in]{geometry}
\usepackage{parcolumns}
\usepackage{fancyvrb}
\usepackage{hyperref}
\usepackage{unicode-math}
\usepackage{graphicx, float}
\usepackage{caption}
\usepackage{adjustbox}

\title{
Отчет по лабораторной работе №3.\\
Реализация системы адаптивного управления
}
\author{Мануилов Георгий, студент гр. 3640102/80201}
\date{}

\begin{document}

\maketitle

\renewcommand\thesection{\arabic{section}}

\section{Постановка задачи}
Требуется разработать систему адаптивного управления в контексте компьютерных сетей. В работе был реализован вариант легенды о зеркалах Архимеда: дано Солнце, которое может менять свое положение; некоторое количество зеркал, которые могут поворачиваться, и точка, в которой требуется сфокусировать солнечные лучи, отраженные зеркалами. 

\section{Технические подробности реализации}
Программа реализована на языке программирования C++. Каждый из агентов является отдельным процессом, взаимодействие между которыми осуществляется с помощью очередей сообщений посредством библиотеки Boost.Interprocess.
\newline
\newline
Программа реализует систему из взаимодействующих агентов четырех типов:
\begin{itemize}
    \item {\bf ControlNode} - агент, организующий работу агентов-исполнителей.
    \item {\bf WorkerNode} - агент-исполнитель, непосредственно занимающийся фокусировкой своего зеркала.
    \item {\bf SunNode} - агент, представляющий Солнце.
    \item {\bf FocusNode} - агент, представляющий точку фокуса.
\end{itemize}
\newline
Для упрощения программы были приняты следующие допущения:
\begin{itemize}
    \item Каналы между агентами являются надежными, то есть сообщение между ними не требует использования протокола автоматического запроса повторной передачи.
    \item Сеть стабильна, то есть ее топология не может поменяться с течением времени.
\end{itemize}

Исходный код решения, инструкции по сборке и запуску, а также примеры представлены в git репозитории по ссылке \url{https://github.com/dev0x13/networks_labs/tree/master/lab3}.

\section{Высокоуровневый алгоритм функционирования системы}
\begin{enumerate}
    \item {\bf ControlNode} устанавливает топологию сети так же, как это происходило в лабораторной работе №2 за исключением поиска кратчайшего пути.
    \item {\bf ControlNode} валидирует топологию сети: требуется, чтобы это была линия.
    \item {\bf ControlNode} находит крайний узел {\bf WorkerNode}, имеющий только одного соседа, и посылает ему сообщение {\it INVOKE}, запуская тем самым непрерывный процесс адаптивной фокусировки.
    \item {\bf WorkerNode}, получивший сообщение, получает координаты Солнца и передает информацию об отраженном луче {\bf FocusNode}.
    \item {\bf FocusNode} сверяет информацию со своими координатами и дает обратную связь в виде значения интенсивности света в фокусе: если отраженный луч попал в фокус, значение будет увеличено.
    \item {\bf WorkerNode} получает значение интенсивности в фокусе и, в случае если значение не увеличилось с предыдущей итерации, поворачивает свое зеркало на небольшой угол и отправляет {\bf FocusNode} новую информацию об отраженном луче. Процедура повторяется до достижения фокусировки.
    \item После фокусировки {\bf WorkerNode} посылает сообщение {\it INVOKE} своему соседу (либо отличного от того, от которого пришло предудщее сообщение {\it INVOKE}, либо тому, от которого пришло сообщение в случае если он является последним в цепи). Таким образом, процесс фокусировки непрерывен и зациклен. Это необходимо, так как координаты {\bf SunNode} постоянно меняются вне зависимости от действий других агентов.
\end{enumerate}
Типы сообщений, используемых агентами для взаимодействия по приведенным выше протоколам и их описания представлены в таблице 1.

\begin{table}[h]
\begin{adjustbox}{width=1.2\textwidth,center}
\begin{tabular}{ |c|c|c|c| }
 \hline
 {\bf Отправитель} & {\bf Получатель} & {\bf Тип сообщени}я & {\bf Описание} \\ [0.5ex] 
 \hline
 WorkerNode & SunNode & PING & Запрос на получение координат Солнца \\ 
 WorkerNode & ControlNode & TOPOLOGY\_OPERATION & Сообщение об изменении топологии сети \\ 
 WorkerNode & FocusNode & VECTOR\_AND\_POINT & Координаты WorkerNode + отраженный вектор \\
 WorkerNode & FocusNode & PING & Запрос на получение интенсивности в фокусе \\
 WorkerNode & WorkerNode & PING & Сообщение ping \\ 
 WorkerNode & WorkerNode & PONG & Ответ на сообщение ping \\ 
 WorkerNode & WorkerNode & INVOKE & Сообщение о передаче потока управления \\ 
 ControlNode & WorkerNode & TOPOLOGY\_OPERATION & Сообщение об изменении топологии сети \\ 
 ControlNode & WorkerNode & INVOKE & Сообщение о передаче потока управления \\ 
 SunNode & WorkerNode & VECTOR & Текущие координаты Солнца \\ 
 FocusNode & WorkerNode & INTENSITY & Текущая интесивность в фокусе \\ 
 \hline
\end{tabular}
\end{adjustbox}
\captionsetup{justification=centering}
\caption{Типы сообщений, используемые агентами для взаимодействия}
\end{table}

\newpage
\section{Примеры работы программы}
Пример конфигурация системы можно найти в папке репозитория \url{https://github.com/dev0x13/networks_labs/tree/master/lab3/presets}. Процессы запускаются асинхронно. Солнце изменяет свои координаты по оси абсцисс от -4 до 4. Приращение координаты - 1 единица в секунду. Конфигурация примера представлена на рис. 1. Каждый из агентов имеет безусловное время жизни, поэтому после нескольких циклов фокусировки все агенты завершают свою работу.

\begin{figure}[!htb]
     \centering
     \includegraphics[width=.4\linewidth]{img.png}
     \caption{Конфигурация примера из папки {\it presets/}}\label{Fig:Data1}
\end{figure}

\newline
\newpage
Вывод примера:
\begin{verbatim}
george@george:~/networks_labs/lab3/presets$ ./run.sh
[WorkerNode r1] New neighbour discovered: `r2`
[ControlNode] Received topology update from DR_r1_backward
[WorkerNode r2] New neighbour discovered: `r1`
[WorkerNode r3] Received topology update from control node
[WorkerNode r3] New neighbour discovered: `r2`
[WorkerNode r2] New neighbour discovered: `r3`
[ControlNode] Received topology update from DR_r3_backward
[WorkerNode r1] Received topology update from control node
[WorkerNode r4] Received topology update from control node
[WorkerNode r4] Received topology update from control node
[WorkerNode r3] New neighbour discovered: `r4`
[ControlNode] Received topology update from DR_r3_backward
[WorkerNode r4] New neighbour discovered: `r3`
[WorkerNode r1] Received topology update from control node
[WorkerNode r2] Received topology update from control node
[ControlNode] Topology researched, determining invocation order
[WorkerNode r4] Invoked by control node
[ControlNode] Terminated
[WorkerNode r4] Focused for Sun coord (-0.7, 5) after 40 rotations
[WorkerNode r3] Invoked by neighbour `r4`
[WorkerNode r3] Focused for Sun coord (-0.8, 5) after 7 rotations
[WorkerNode r2] Invoked by neighbour `r3`
[WorkerNode r2] Focused for Sun coord (-0.8, 5) after 1 rotations
[WorkerNode r1] Invoked by neighbour `r2`
[WorkerNode r1] Focused for Sun coord (-1.2, 5) after 14 rotations
[WorkerNode r2] Invoked by neighbour `r1`
[WorkerNode r2] Focused for Sun coord (-1.3, 5) after 2 rotations
[WorkerNode r3] Invoked by neighbour `r2`
[WorkerNode r3] Focused for Sun coord (-1.3, 5) after 21 rotations
[WorkerNode r4] Invoked by neighbour `r3`
[WorkerNode r4] Focused for Sun coord (-1.4, 5) after 35 rotations
[WorkerNode r3] Invoked by neighbour `r4`
[WorkerNode r3] Focused for Sun coord (-1.4, 5) after 17 rotations
[WorkerNode r2] Invoked by neighbour `r3`
[WorkerNode r2] Focused for Sun coord (-1.5, 5) after 16 rotations
[WorkerNode r1] Invoked by neighbour `r2`
[WorkerNode r1] Focused for Sun coord (-1.6, 5) after 7 rotations
[WorkerNode r2] Invoked by neighbour `r1`
[WorkerNode r2] Focused for Sun coord (-1.6, 5) after 16 rotations
[WorkerNode r3] Invoked by neighbour `r2`
[WorkerNode r3] Focused for Sun coord (-1.7, 5) after 13 rotations
[WorkerNode r4] Invoked by neighbour `r3`
[WorkerNode r4] Focused for Sun coord (-1.9, 5) after 15 rotations
[WorkerNode r3] Invoked by neighbour `r4`
[WorkerNode r3] Focused for Sun coord (-1.9, 5) after 17 rotations
[WorkerNode r2] Invoked by neighbour `r3`
[WorkerNode r2] Focused for Sun coord (-2.1, 5) after 17 rotations
[WorkerNode r1] Invoked by neighbour `r2`
[WorkerNode r1] Focused for Sun coord (-2.2, 5) after 16 rotations
[SunNode] Terminated
[FocusNode] Terminated
[WorkerNode r1] Terminated
[WorkerNode r2] Terminated
[WorkerNode r3] Terminated
[WorkerNode r4] Terminated

\end{verbatim}

На рис. 2 показан пример процесса фокусировки зеркала с координатами (3, 0) в фокусе с координатами (2.5, 1) в момент, когда Солнце находится в точке с координатами (-3.9, 5). В таблице 2 представлены соответствующие значения координат вектора нормали и отраженного вектора.

\begin{figure}[!htb]
     \centering
     \includegraphics[width=.8\linewidth]{img2.png}
     \caption{Иллюстрация процесса фокусировки зеркала}\label{Fig:Data1}
\end{figure}

\begin{table}[h]
\centering
\begin{tabular}{ |c|c|c|c| }
 \hline
 {\bf Отраженный вектор} & {\bf Вектор нормали} \\ [0.5ex] 
 \hline
 (4.82641, 7.02253) & (0.985451, 0.169967) \\
 (3.33504, 7.84141) & (0.963559, 0.267499) \\
 (1.71071, 8.34767) & (0.93204, 0.362358) \\
 (0.0181866, 8.52114) & (0.891208, 0.453596) \\
 (-1.67507, 8.35489) & (0.841472, 0.540303) \\
 (-3.30154, 7.85556) & (0.783328, 0.621611) \\
 \hline
\end{tabular}
\captionsetup{justification=centering}
\caption{Значения координат вектора нормали и отраженного вектора}
\end{table}

\end{document}

