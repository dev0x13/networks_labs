\documentclass{report}
\usepackage[T1,T2A]{fontenc}
\usepackage[utf8]{inputenc}
\usepackage[russian]{babel}
\usepackage[a4paper, margin=1in]{geometry}
\usepackage{parcolumns}
\usepackage{fancyvrb}
\usepackage{hyperref}
\usepackage{unicode-math}
\usepackage{graphicx, float}
\usepackage{caption}


\title{
Отчет по лабораторной работе №2.\\
Реализация протокола динамической маршрутизации Open Shortest Path First
}
\author{Мануилов Георгий, студент гр. 3640102/80201}
\date{}

\begin{document}

\maketitle

\renewcommand\thesection{\arabic{section}}

\section{Постановка задачи}
Требуется разработать систему из неограниченного количества взаимодействующих друг с другом маршрутизаторов, которые орагнизуются в сеть и обеспечивают передачу сообщений от каждого маршрутизатора к каждому по кратчайшему пути.

\section{Технические подробности реализации}
Программа-маршрутизатор реализована на языке программирования C++. Программа может функционировать в двух режимах: в режиме обычного маршрутизатора и в режиме выделенного маршрутизатора (Designated Router). Сеть задается конфигурацией множества обычных маршрутизаторов с явным указанием смежных маршрутизаторов и конфигурацией одного выделенного маршрутизатора. Для упрощения реализации сеть не содержит резервного выделенного маршрутизатора. Взаимодействие между процессами осуществляется с помощью очередей сообщений посредством библиотеки Boost.Interprocess.
\newline
\newline
Программа разделена на следующие сущности:
\begin{itemize}
    \item {\bf OneWayTransducer} - примитивный приемник/передатчик. Не контролирует доставку, работает только в одну сторону, утилизирует одну очередь сообщений.
    \item {\bf Topology} - класс, описывающий топологию сети. По сути представляет из себя обертку над графом с набором высокоуровневых методов.
    \item {\bf DesignatedRouter} - выделенный маршрутизатор. Хранит внутри себя экземпляр топологии сети. Обрабатывает сообщения маршрутизаторов об изменении топологии сети и транслирует их остальным маршрутизаторам.
    \item {\bf CommonRouter} - обычный маршрутизатор. Хранит внутри себя экземпляр топологии сети. После запуска начинает непрерывный обмен сообщениями с соседями. При появлении нового соседа или потере старого посылает сообщение об изменении топологии выделенному маршрутизатору. При получении сообщения от выделенного маршрутизатора вносит изменения в свой экземпляр топологии сети и перестраивает кратчайшие маршруты от себя до каждого узла в сети с помощью алгоритма Дейкстры.
\end{itemize}
\newline
Программа-маршрутизатор получает на вход путь к конфигурационному файлу, который содержит в себе следующие поля:
\begin{itemize}
    \item {\bf id} - уникальный идентификатор маршрутизатора
    \item {\bf neighbours} - список смежных маршрутизаторов и стоимость каналов связи с ними
\end{itemize}
\newline
Исходный код решения, инструкции по сборке и запуску, а также примеры представлены в git репозитории по ссылке \url{https://github.com/dev0x13/networks_labs/tree/master/lab2}.
\newpage
\section{Примеры работы программы}
Примеры конфигурация сети можно найти в папке репозитория \url{https://github.com/dev0x13/networks_labs/tree/master/lab2/presets}. Процессы запускаются асинхронно. Процесс выделенного маршрутизатора после завершения (завершение происходит, если в течение 5 секунд от маршрутизаторов не поступило ни одного сообщения об изменении топологии) сохраняет текущую топологию сети в файл DOT, который затем можно визуализировать с помощью утилиты graphviz (визуализации для примеров приведены на рис. 1, 2, 3).
\newline
\newline
Вывод примера "cross":
\begin{verbatim}
$ ./router cfg/cross/dr.cfg 
[DesignatedRouter] Received topology update from DR_r1_receive
[DesignatedRouter] Received topology update from DR_r2_receive
[DesignatedRouter] Received topology update from DR_r3_receive
[DesignatedRouter] Received topology update from DR_r3_receive
[DesignatedRouter] No topology updates during 5000 ms, terminating
\end{verbatim}
\begin{verbatim}
$ ./router cfg/cross/r1.cfg 
[CommonRouter r1] New neighbour discovered: `r3`
[CommonRouter r1] Received topology update from DR
[CommonRouter r1] Shortest paths found
[CommonRouter r1] Received topology update from DR
[CommonRouter r1] Shortest paths found
[CommonRouter r1] Received topology update from DR
[CommonRouter r1] Shortest paths found
[CommonRouter r1] Neighbour `r3` is dead
[CommonRouter r1] No topology updates during 2000 ms, terminating
\end{verbatim}
\begin{verbatim}
$ ./router cfg/cross/r2.cfg 
[CommonRouter r2] New neighbour discovered: `r3`
[CommonRouter r2] Received topology update from DR
[CommonRouter r2] Shortest paths found
[CommonRouter r2] Received topology update from DR
[CommonRouter r2] Shortest paths found
[CommonRouter r2] Received topology update from DR
[CommonRouter r2] Shortest paths found
[CommonRouter r2] Neighbour `r3` is dead
[CommonRouter r2] No topology updates during 2000 ms, terminating
\end{verbatim}
\begin{verbatim}
$ ./router cfg/cross/r3.cfg 
[CommonRouter r3] Received topology update from DR
[CommonRouter r3] Shortest paths found
[CommonRouter r3] Received topology update from DR
[CommonRouter r3] Shortest paths found
[CommonRouter r3] Neighbour `r1` is dead
[CommonRouter r3] Neighbour `r2` is dead
[CommonRouter r3] New neighbour discovered: `r1`
[CommonRouter r3] New neighbour discovered: `r2`
[CommonRouter r3] New neighbour discovered: `r4`
[CommonRouter r3] New neighbour discovered: `r5`
[CommonRouter r3] No topology updates during 2000 ms, terminating
\end{verbatim}
\begin{verbatim}
$ ./router cfg/cross/r4.cfg 
[CommonRouter r4] Received topology update from DR
[CommonRouter r4] Shortest paths found
[CommonRouter r4] Received topology update from DR
[CommonRouter r4] Shortest paths found
[CommonRouter r4] Received topology update from DR
[CommonRouter r4] Shortest paths found
[CommonRouter r4] Neighbour `r3` is dead
[CommonRouter r4] New neighbour discovered: `r3`
[CommonRouter r4] Received topology update from DR
[CommonRouter r4] Shortest paths found
[CommonRouter r4] Neighbour `r3` is dead
[CommonRouter r4] No topology updates during 2000 ms, terminating
\end{verbatim}
\begin{verbatim}
$ ./router cfg/cross/r5.cfg 
[CommonRouter r5] Received topology update from DR
[CommonRouter r5] Shortest paths found
[CommonRouter r5] Received topology update from DR
[CommonRouter r5] Shortest paths found
[CommonRouter r5] Received topology update from DR
[CommonRouter r5] Shortest paths found
[CommonRouter r5] New neighbour discovered: `r3`
[CommonRouter r5] Neighbour `r3` is dead
[CommonRouter r5] No topology updates during 2000 ms, terminating

\end{verbatim}
\begin{figure}[!htb]
   \begin{minipage}{0.3\textwidth}
     \centering
     \includegraphics[width=.9\linewidth]{cross.png}
     \caption{Пример "cross"}\label{Fig:Data1}
   \end{minipage}\hfill
   \begin{minipage}{0.3\textwidth}
     \centering
     \includegraphics[width=.9\linewidth]{circle.png}
     \caption{Пример "circle"}\label{Fig:Data2}
   \end{minipage}\hfill
   \begin{minipage}{0.3\textwidth}
     \centering
     \includegraphics[width=.9\linewidth]{full.png}
     \caption{Пример "full"}\label{Fig:Data2}
   \end{minipage}
\end{figure}
\end{document}

